\documentclass[11pt]{article}

\usepackage{extras} % Se extras.sty

\begin{document}

\begin{titlepage}
\begin{center}

{\Large\bfseries TSEA56 - Kandidatprojekt i elektronik \\ LIPS Användarhandledning}

\vspace{5em}

Version 1.0

\vspace{5em}
Grupp 4 \\
\begin{tabular}{rl}
Hynén Ulfsjöö, Olle&\verb+ollul666+
\\
Wasteson, Emil&\verb+emiwa068+
\\
Tronje, Elena&\verb+eletr654+
\\
Gustafsson, Lovisa&\verb+lovgu777+
\\
Inge, Zimon&\verb+zimin415+
\\
Strömberg, Isak&\verb+isast763+
\\
\end{tabular}

\vspace{5em}
\today

\vspace{16em}
Status
\begin{longtable}{|l|l|l|} \hline

Granskad & - & - \\ \hline
Godkänd & - & - \\ \hline
 
\end{longtable}

\end{center}
\end{titlepage}

\pagebreak
\begin{center}

\section*{PROJEKTIDENTITET}
2016/VT, Undsättningsrobot Gr. 4

Linköpings tekniska högskola, ISY
\vspace{5em}

\begin{tabular}{|l|l|l|l|} \hline
\textbf{Namn} & \textbf{Ansvar} & \textbf{Telefon} & \textbf{E-post}  \\ \hline 
Isak Strömberg (IS) & Projektledare & 073-980 38 50 & isast763@student.liu.se \\ \hline
Olle Hynén Ulfsjöö (OHU)& Dokumentansvarig & 070-072 91 84 & ollul666@student.liu.se \\ \hline
Emil Wasteson (EW) & Hårdvaruansvarig & 076-836 61 66 & emiwa068@student.liu.se \\ \hline
Elena Tronje (ET) & Mjukvaruansvarig & 072-276 92 93 & eletr654@student.liu.se \\ \hline
Zimon Inge (ZI) & Testansvarig & 070-171 35 18 & zimin415@student.liu.se \\ \hline
Lovisa Gustafsson (LG) & Leveransansvarig & 070-210 32 53 & lovgu777@student.liu.se \\ \hline
\end{tabular}


E-postlista för hela gruppen: isast763@student.liu.se

\vspace{5em}
Kund: ISY, Linköpings universitet 
tel: 013-28 10 00, fax: 013-13 92 82 \\
Kontaktperson hos kund: Mattias Krysander \\
tel: 013-28 21 98, e-post: matkr@isy.liu.se \\

\vspace{5em}
Kursansvarig:  Tomas Svensson\\
tel: 013-28 13 68, e-post: tomass@isy.liu.se \\
Handledare: Peter Johansson \\
tel: 013-28 13 45, e-post: peter.a.johansson@liu.se
\end{center}
\pagebreak

\tableofcontents

\pagebreak

\section*{Dokumenthistorik}
\begin{table}[h]
\begin{tabular}{|l|l|l|l|l|} \hline

\textbf{Version} & \textbf{Datum} & \textbf{Utförda förändringar} & \textbf{Utförda av} & \textbf{Granskad} \\ \hline
1.0 & - & Första utkastet & Grupp 4 & - \\ \hline
\end{tabular}
\end{table}

\pagebreak
\section{Inledning}
Undsättningsroboten PigBot är en robot för undsättning av nödställda. Den är autonom konstruktion vilket innebär att PigBot självständigt kan navigera, och identifiera nödställda, i en labyrint utan att användarens ingripanden krävs.

Denna användarhandledning har för avsikt att ingående instruera användaren vilka användningsområden undsättningsroboten, PigBot, besitter. Här ges en ingående beskrivning hur PigBot kan kopplas samman med en extern datormodul, och hur det grafiska gränssnittet på datormodulen fungerar och bör användas. Andarhandledningen ska dessutom demonstrera hur hårdvåran på PigBot fungerar, detta ur ett användarmässigt perspektiv. 

För att bruka PigBot krävs en dator med möjligthet till parning via Bluetooth\textsuperscript{\circledR}. Om användaren förfogar över en dator som inte har denna möjlighet krävs dessutom en Bluetooth\textsuperscript{\circledR}-adapter för att kunna ansluta till PigBot.

\subsection{Förkortningar}
Denna användarhandledning kommer tillämpa förkortningar för att ge användaren bättre översikt i sin handledning. Samtliga förkortningar är emellertid listade nedan:

\begin{itemize}
\item GUI - Står för "graphical user interface" vilket är engelska för grafiskt användargränssnitt. Detta är den mjukvara som ska underlätta interaktionen mellan användaren och PigBot.
\end{itemize}

\section{Roboten}
Nedan följer en användarhandledning för själva roboten. Kapitlet är uppdelat i två delar. Dioder och brytare beskriver de fysiska komponenter som är monterade på roboten. Styrning ger en förklaring till hur roboten ska användas i autonomt respektive manuellt läge.

\subsection{Dioder och brytare}
Hädanefter numreras dioder och brytare enligt figur \ref{picTop} och nedan följer en förklaring av respektive.

De dioder som är placerade på robotens övre modul visar båda styrläge och status under uppdraget.
\begin{description}[style=unboxed, leftmargin=0cm]
  \item[Diod 1] visar ifall roboten befinner sig i autonomt eller manuellt läge. 
    \begin{itemize}
      \setlength\itemsep{-0.5em}
      \item[-] På: Autonomt läge 
      \item[-] Av: Manuellt läge
    \end{itemize}
  \item[Diod 2] visar ifall roboten befinner sig i \textit{debug}-läge läge eller inte.
    \begin{itemize}
      \setlength\itemsep{-0.5em}
      \item[-] På: \textit{debug}-läge
      \item[-] Av: Vanligt läge
    \end{itemize}
  \item[Diod 3] visar ifall roboten är klar med uppdraget eller inte.
    \begin{itemize}
      \setlength\itemsep{-0.5em}
      \item[-] På: Roboten är klar med uppdraget
      \item[-] Av: Roboten är ännu inte klar med uppdraget
    \end{itemize}
  \item[Diod 4 och 5] visar tillsammans statusen för roboten under uppdragets genomförande. Dioderna räknas upp binärt i takt med att ett moment har avklarats. Diod 5 är minst signifikant bit. De möjliga kombinationer som dioderna kan anta och dess beskrivning återfinns nedan.
    \hspace{1em}
    \begin{tabular}{c c p{6cm}}
	Diod 5 & Diod 6 & Beskrivning \\ \hline
	Av & Av & Väntar på startkommando \\
	Av & På & Söker efter målet \\
	På & Av & Har funnit målet och återkommit till startposition. Gripklon inväntar en förnödenhet. \\
	På & På & Klon är stängd och roboten är redo för att färdas kortaste vägen till målet.

    \end{tabular}
\end{description}


\subsection{Styrning}


\section{Datormodul}
\subsection{Autonomt läge}
\subsection{Manuellt läge}

\pagenumbering{arabic}

\end{document}
