\documentclass[11pt]{article}

\usepackage{extras} % Se extras.sty

\begin{document}
\begin{titlepage}
\begin{center}

{\Large\bfseries TSEA56 - Kandidatprojekt i elektronik \\ LIPS Teknisk dokumentation}

\vspace{5em}

Version 1.0

\vspace{5em}
Grupp 4 \\
\begin{tabular}{rl}
Hynén Ulfsjöö, Olle&\verb+ollul666+
\\
Wasteson, Emil&\verb+emiwa068+
\\
Tronje, Elena&\verb+eletr654+
\\
Gustafsson, Lovisa&\verb+lovgu777+
\\
Inge, Zimon&\verb+zimin415+
\\
Strömberg, Isak&\verb+isast763+
\\
\end{tabular}

\vspace{5em}
\today

\vspace{16em}
Status
\begin{longtable}{|l|l|l|} \hline

Granskad & - & - \\ \hline
Godkänd & - & - \\ \hline
 
\end{longtable}

\end{center}
\end{titlepage}

\pagebreak
\begin{center}

\section*{PROJEKTIDENTITET}
2016/VT, Undsättningsrobot Gr. 4

Linköpings tekniska högskola, ISY
\vspace{5em}
%\begin{center}

\begin{tabular}{|l|l|l|l|} \hline
\textbf{Namn} & \textbf{Ansvar} & \textbf{Telefon} & \textbf{E-post}  \\ \hline 
Isak Strömberg (IS) & Projektledare & 073-980 38 50 & isast763@student.liu.se \\ \hline
Olle Hynén Ulfsjöö (OHU)& Dokumentansvarig & 070-072 91 84 & ollul666@student.liu.se \\ \hline
Emil Wasteson (EW) & Hårdvaruansvarig & 076-836 61 66 & emiwa068@student.liu.se \\ \hline
Elena Tronje (ET) & Mjukvaruansvarig & 072-276 92 93 & eletr654@student.liu.se \\ \hline
Zimon Inge (ZI)& Testansvarig & 070-171 35 18 & zimin415@student.liu.se \\ \hline
Lovisa Gustafsson (LG) & Leveransansvarig & 070-210 32 53 & lovgu777@student.liu.se \\ \hline
\end{tabular}

%\end{center}

E-postlista för hela gruppen: isast763@student.liu.se

\vspace{5em}
Kund: ISY, Linköpings universitet 
tel: 013-28 10 00, fax: 013-13 92 82 \\
Kontaktperson hos kund: Mattias Krysander \\
tel: 013-28 21 98, e-post: matkr@isy.liu.se \\

\vspace{5em}
Kursansvarig:  Tomas Svensson\\
tel: 013-28 13 68, e-post: tomass@isy.liu.se \\
Handledare: Peter Johansson \\
tel: 013-28 13 45, e-post: peter.a.johansson@liu.se
\end{center}
\pagebreak

\tableofcontents

\pagebreak

\section*{Dokumenthistorik}
\begin{table}[h]
\begin{tabular}{|l|l|l|l|l|} \hline

\textbf{Version} & \textbf{Datum} & \textbf{Utförda förändringar} & \textbf{Utförda av} & \textbf{Granskad} \\ \hline
1.0 & - &  Första utkastet & Grupp 4 & - \\ \hline
\end{tabular}
\end{table}

\pagebreak
\pagenumbering{arabic}

\begin{flushleft}
\section{Inledning}
\textit{Bakgrund och syfte}

Projektets syfte är att konstruera en undsättningsrobot med kartläggning. Detta dokument syftar till att ge en detaljerad teknisk beskrivning av projektet på en nivå där resultatet ska kunna återskapas. Innehållet består av detaljerade beskrivningar av systemets olika delar, dess mjuk- och hårdvarukomponenter samt implementationen av dessa.

\section{Produkten}
\textit{En bild på produkten och en beskrivning av hur den fungerar. Beskriv vad den används till.}

Produkten som konstruerats är en undsättningsrobot. Den består av ett chassi med fyra hjul, ett batteri, en av/på-brytare samt en gripklo. På chassit är huvudmodulen, sensormodulen och styrmodulen ihopkopplade via en I\textsuperscript{2}C-buss för kommunikation dem emellan. Roboten kommunicerar med datormodulen via Bluetooth\textsuperscript{\circledR}. En övergripande bild av produkten återfinns i figur \ref{overview}. Gripklon kontrolleras av styrmodulen och sensormodulen tar in data från sensorerna. För att fästa sensorerna på chassit används 3D-printade fästen.

Varje modul har en processor samt annan nödvändig hårdvara, som lågpassfilter och brytare, kopplade på ett virkort. Virkorten kan användas som grund för att tillverka kretskort med motsvarande funktion. 

Robotens uppgift är att söka av en labyrint bestående av 40 cm breda korridorer med hjälp av en avsökningsalgoritm och identifiera en nödställd som sänder ut IR-ljus. När målet är identifierat ska kortaste vägen mellan labyrintens ingång och målet bestämmas. Efter det ska roboten hämta en förnödenhet vid starten, köra kortaste vägen till den nödställda och lämna av förnödenheten för att sedan ta sig tillbaka samma väg till starten. Roboten kartlägger labyrinten internt under tiden den utforskar den och lagrar informationen för att kunna skicka den till datormodulen för uppritning. 


\begin{figure}[!htbp]
\centering
\noindent\resizebox{\linewidth}{!}{
	\documentclass{minimal}
\usepackage{tikz}
%\usetikzlibrary{calc,trees,positioning,arrows,chains,shapes.geometric,decorations.pathreplacing,decorations.pathmorphing,shapes,matrix,shapes.symbols}
\usetikzlibrary{positioning}
\usetikzlibrary{shapes}

\begin{document}
\begin{center}
\begin{tikzpicture}[scale=1]
\tikzset{every node/.style={inner sep=10pt, minimum width=3 cm}}
%\draw[help lines,step=5mm,gray!20] (-5,-10) grid (5,0);
\node[draw, fill=white] (Huvudmodul)  {\textbf{Huvudmodul}};
\node[draw,below left= of Huvudmodul] (Sensormodul) {\textbf{Sensormodul}};
\node[draw,below right = of Huvudmodul] (Styrenhet) {\textbf{Styrenhet}};
\node[draw, above = of Huvudmodul] (Datormodul) {\textbf{Datormodul}};
\node[ellipse,draw, right = of Datormodul] (Användare) {\textbf{Användare}};

\draw[->] (Huvudmodul) [out=300, in= 90] to (Styrenhet);
\draw[->] (Sensormodul) [out=90, in=240] to (Huvudmodul);
\draw[<->] (Huvudmodul) to (Datormodul);
\draw[<->] (Datormodul) to (Användare);
\end{tikzpicture}
\end{center}

\vspace{10em}

\begin{center}
\begin{tikzpicture}[scale=1]
\tikzset{every node/.style={inner sep=10pt, minimum width=3 cm}}
%\draw[help lines,step=5mm,gray!20] (-5,-10) grid (5,0);

\node[draw] (Sensormodul) {\textbf{Sensormodul}};
\node[above right = of Sensormodul,minimum width=0,inner sep=2pt] (Huvudmodul) {Huvudmodul};

\node[below left = of Sensormodul, minimum width = 0, inner sep = 2pt] (Sensor) {Sensorer};


\draw[->] (Sensor) [out=0,in=270] to node [sloped, midway, below] {spänningsnivåer} (Sensormodul);
\draw[->] (Sensormodul) [out=90,in=180] to node [sloped,midway, above] {enheter}  (Huvudmodul);

\end{tikzpicture}
\end{center}

\vspace{10em}

\begin{center}
\begin{tikzpicture}[scale=1]
\tikzset{every node/.style={inner sep=10pt, minimum width=3 cm}}
%\draw[help lines,step=5mm,gray!20] (-5,-10) grid (5,0);

\node[draw] (Styrmodul) {\textbf{Styrmodul}};
\node[above left = of Styrmodul,minimum width=0,inner sep=2pt] (Huvudmodul) {Huvudmodul};

\node[below right = of Styrmodul, minimum width = 0, inner sep = 2pt] (Motorer) {Motorer};
%\node[below left = of Styrmodul, minimum width = 0, inner sep= 2pt] (Gripklo) {Gripklo}


\draw[->] (Styrmodul) [out=270,in=180] to node [sloped, midway, below] {spänningsnivåer} (Motorer);
\draw[<-] (Sensormodul) [out=90,in=0] to node [sloped,midway, above] {kommandon}  (Huvudmodul);

\end{tikzpicture}
\end{center}

\vspace{10em}

\begin{center}
\begin{tikzpicture}[scale=1]
\tikzset{every node/.style={inner sep=10pt, minimum width=3 cm}}
%\draw[help lines,step=5mm,gray!20] (-5,-10) grid (5,0);

\node[draw] (Datormodul) {\textbf{Datormodul}};
\node[below = 10 em of Datormodul,minimum width=0,inner sep=2pt] (Huvudmodul) {Huvudmodul};
\node[right = of Datormodul, ellipse, draw] (Användare) {Användare};

%\node[below right = of Styrmodul, minimum width = 0, inner sep = 2pt] (Motorer) {Motorer};
%\node[below left = of Styrmodul, minimum width = 0, inner sep= 2pt] (Gripklo) {Gripklo}


\draw[->] (Datormodul) [out=30,in=150] to node [midway, above] {karta} (Användare);
\draw[<-] (Datormodul) [out=-30,in=210] to node [midway, below] {kommandon}  (Användare);

\draw[->] (Datormodul) [out=300,in=60] to node [near end, right] {kommandon} (Huvudmodul);
\draw[<-] (Datormodul) [out=240,in=120] to node [near end, left] {sensordata} (Huvudmodul);

\end{tikzpicture}
\end{center}


\end{document}}
	\caption{Det totala systemet \label{overview}}	
\end{figure}

\pagebreak


\section{Teori}
\textit{Beskrivning av regleralgoritmer mm.}

\section{Systemet}
\textit{Ett översiktligt blockschema och en beskrivning av hela systemet.}

Roboten i sin miljö finns illustrerad i figur \ref{system}. Kommunikationen med datormodulen är dubbelriktad och via Bluetooth\textsuperscript{\circledR}. Roboten ska dock även klara sitt uppdrag utan kommunikation med datormodulen. Det vill säga att kartläggning, styrning och optimering av kortaste väg sker lokalt hos roboten. Banan är uppbyggd enligt banspecifikationen och uppdraget utförs enligt tävlingsreglerna, se appendix [??].

\begin{figure}[htbp]
\centering
\noindent\resizebox{.8\linewidth}{!}{
	\input{images/bana}}
	\caption{Översikt av banan\label{system}}	
\end{figure}

\subsection{Beskrivning av systemet}
Roboten navigerar själva banan med hjälp av en lasersensor som mäter avstånd framåt, fyra IR-sensorer som mäter robotens avstånd till respiktive vägg i korridoren, ett gyroskop som mäter vinkelhastighet samt en IR-detektor för identifiering av nödställd. En regleringsmodell säkerställer att roboten färdas i mitten av korridorerna samt kan rotera både 90 och 180 grader utan att stöta mot väggar. Under färden ska roboten autonomt kartlägga och finna den kortaste vägen mellan ingången och den nödställde. I det fall då roboten även är uppkopplad mot datormodulen ska mjukvara på datorn successivt rita upp en karta och kunna presentera utvalda mätvärden i realtid. När den kortaste vägen är funnen ska roboten använda denna rutt för att förse den nödställde med en förnödenhet. Förnödenheten transporteras med hjälp av robotens gripklo.

Blockshemat för systemet återfinns i figur \ref{blockSystem}.

\begin{figure}[htbp]
\centering
\noindent\resizebox{1\linewidth}{!}{
	\documentclass[border=10px]{standalone}
\usepackage{tikz}
\usetikzlibrary{patterns}
\usetikzlibrary{shapes.geometric}
\usetikzlibrary{shapes.arrows}
\usepackage{amssymb}
\usetikzlibrary{calc}
\usepackage{verbatim}

\pagestyle{empty}
\begin{document}

\tikzstyle{decision} = [diamond, draw,
    text width=4em, text badly centered, node distance=3cm, inner sep=0pt]
    
\tikzstyle{block} = [rectangle, draw,
    text width=5em, text centered, rounded corners, minimum height=4em]
	
\begin{tikzpicture}[scale=1]

%http://www.texample.net/tikz/examples/simple-flow-chart/

\node[block](start){Start};
\node[decision, below of = start, aspect = 2, text width = 8 em] (foundTarget) {Målet funnet?};
\node[block, below of = foundTarget, node distance = 3cm, text width = 7em] (explore) {Utforska enligt högerföljning};
\node[decision, right of = foundTarget, text width = 6em, aspect = 2, node distance = 6cm] (foundShortestPath) {Garanterat funnit kortaste väg?};
\node[block, below of = foundShortestPath, node distance = 3cm, text width = 7em] (exploreTargetFound) {Utforska enligt högerföljning och optimering};
\node[block, right of = foundShortestPath, node distance = 5cm] (pickUp) {Hämta förnödenhet vid start};
\node[block, right of = pickUp, node distance = 3cm, text width = 6em] (shortestPath) {Från start följ kortaste väg till målet};
\node[block, right of = shortestPath, node distance = 3cm] (drop) {Lämna förnödenhet};
\node[block, below of = drop, node distance = 3cm, text width = 7em] (return) {Återvänd till startpunkt den kortaste vägen};
\node[block, below of = return, node distance = 3cm] (stop) {Slut};

\draw[->](start) --  (foundTarget);
\draw[->](foundTarget) -- node[near start, right]{nej} (explore);
\draw[->] (explore.west) -| ++(-1,3) -| (foundTarget.west);
\draw[->](foundTarget) -- coordinate[midway] (aux) node[near start, above]{ja} (foundShortestPath);
\draw[->] (foundShortestPath) -- node[near start, right]{nej}(exploreTargetFound);
\draw[->] (exploreTargetFound) -| (aux);
\draw[->] (foundShortestPath) -- node[near start, above]{ja}(pickUp);
\draw[->] (pickUp) -- (shortestPath);
\draw[->] (shortestPath) -- (drop);
\draw[->] (drop) -- (return);
\draw[->] (return) -- (stop);

	\end{tikzpicture}
	
\end{document}}
	\caption{Övergripande blockschema för systemet}	\label{blockSystem}
\end{figure}

\section{Begränsningar}
Roboten är begränsad till att navigera i enbart korridorer och att alla korsningar sker med 90 gradersvinklar.

\section{Modulerna}
\textit{Innehåller mera detaljerade blockschemor och beskrivningar av varje modul. Tänk på läsbarheten och växla mellan figurer och text.}
I detta avsnitt följer en detaljerad beskrivning av systemets ingående moduler. 

\subsection{Huvudmodulen}
Huvudmodulen har som uppgift att ta emot och förmedla information mellan de andra modulera, hantera den interna kartläggnigen samt att ta alla övergripande beslut. Bluetooth\textsuperscript{\circledR} används för att kommunicera med datormodulen och en avbrottsstyrd I\textsuperscript{2}C-buss används för att kommunicera med sensor- och styrmodulen, vilket visas i figur \ref{communication}. När det kommer till prioriteringen av sensor- och styrmodul har styrmodulen kopplats till en extern avbrottsingång med högre prioritet än sensormodulen.

\begin{figure}[htbp]
\noindent\resizebox{.97\textwidth}{!}{
	\documentclass[border=20pt]{standalone}
\usepackage{tikz}
\usetikzlibrary{positioning}
\usetikzlibrary{calc}
\usetikzlibrary{decorations.pathmorphing}
\usepackage{amssymb}
\usetikzlibrary{shapes,arrows}

\begin{document}
	\begin{tikzpicture}[scale=1]
		
		\tikzset{every node/.style={thick, draw=black, align=center, minimum height=40pt, text width=100pt, minimum width=100pt}}
		\node(datormodul) {Datormodul};
		\node[right=10pt of datormodul,minimum height=20pt, minimum width=10pt,text width=10pt] (bt1) {\includegraphics{bluetooth}};
		
		\node[right=40pt of bt1,minimum height=20pt, minimum width=10pt,text width=10pt] (bt2) {\includegraphics{bluetooth}};
		
		\node[right=10pt of bt2] 			(huvudmodul) 	{Huvudmodul};
		\node[below=-10pt of huvudmodul,draw=none] (master) {\textit{master}};
		\node[right=10pt of huvudmodul] 		(sensormodul) 	{Sensormodul};
		\node[below=-10pt of sensormodul,draw=none] (slave1) {\textit{slav}};
		\node[right=10pt of sensormodul] 	(styrmodul) 		{Styrmodul};
		\node[below=-10pt of styrmodul,draw=none] (slave2) {\textit{slav}};
		
		\coordinate (sclStart) 	at ($(huvudmodul.north west) + (0,20pt)$);
		\coordinate (sclEnd)		at ($(styrmodul.north east)  + (0,20pt)$);
		
		\coordinate (sdaStart)  at ($(sclStart) + (0,20pt)$);
		\coordinate (sdaEnd)		at ($(sclEnd)	+ (0,20pt)$);
		
		\coordinate (vddStart)  at ($(sdaStart) + (0,40pt)$);
		\coordinate (vddEnd)		at ($(sdaEnd)	+ (0,40pt)$);
		
		\draw[thick] (sclStart) -- (sclEnd) node [right,draw=none,text width=0,minimum width=0] {SCL};
		\draw[thick] (sdaStart) -- (sdaEnd) node [right,draw=none,text width=0,minimum width=0] {SDA};
		\draw[thick] (vddStart) -- (vddEnd) node [right,draw=none,text width=0,minimum width=0] {$V_{dd}$};
		
		\draw[thick] (datormodul.east) -- (bt1.west);
		
		\draw[thick, ->,line join=round,decorate, decoration={
    												snake,
    												segment length=5,
    												amplitude=1,
    												post=lineto,
    												post length=1pt}] 
    		($(bt1.east) + (5pt,5pt)$) -- ($(bt2.west) + (-5pt,5pt)$);
    		
    	\draw[thick, ->,line join=round,decorate, decoration={
    												snake,
    												segment length=5,
    												amplitude=1,
    												post=lineto,
    												post length=1pt}] 
    		 ($(bt2.west) + (-5pt,-5pt)$) -- ($(bt1.east) + (5pt,-5pt)$);
    		 
    	\draw[thick] (bt2.east) -- (huvudmodul.west);
    	
    	\draw[thick,fill=black] ($(huvudmodul.north) + (-10pt,0)$) -- ($(huvudmodul.north) + (-10pt,20pt)$) circle [radius=2pt];
    	\draw[thick,fill=black] ($(huvudmodul.north) + (+10pt,0)$) -- ($(huvudmodul.north) + (+10pt,40pt)$) circle [radius=2pt];
    	
    	\draw[thick,fill=black] ($(sensormodul.north) + (-10pt,0)$) -- ($(sensormodul.north) + (-10pt,20pt)$) circle [radius=2pt];
    	\draw[thick,fill=black] ($(sensormodul.north) + (+10pt,0)$) -- ($(sensormodul.north) + (+10pt,40pt)$) circle [radius=2pt];
    	
    	\draw[thick,fill=black] ($(styrmodul.north) + (-10pt,0)$) -- ($(styrmodul.north) + (-10pt,20pt)$) circle [radius=2pt];
    	\draw[thick,fill=black] ($(styrmodul.north) + (+10pt,0)$) -- ($(styrmodul.north) + (+10pt,40pt)$) circle [radius=2pt];
    	
    	\draw[thick,fill=black] ($(sensormodul.north east) + (-5pt,20pt)$) circle [radius=2pt] -- ($(sensormodul.north east) + (-5pt,80pt)$) circle [radius=2pt];
    	\draw[thick,fill=black] ($(styrmodul.north west) + (5pt,40pt)$) circle [radius=2pt] -- ($(styrmodul.north west) + (5pt,80pt)$) circle [radius=2pt];
    	\draw[thick,draw=black,fill=white] ($(styrmodul.north west) + (2pt,50pt)$) rectangle ($(styrmodul.north west) + (8pt,70pt)$);
		\draw[thick,draw=black,fill=white] ($(sensormodul.north east) + (-8pt,50pt)$) rectangle ($(sensormodul.north east) + (-2pt,70pt)$) node [below left=-10pt and 20pt,draw=none,minimum width=0, text width = 0pt] {$R_p$};
		
		\draw[thick, ->] ($(sensormodul.south) + (+40pt,0)$) -- ($(sensormodul.south) + (40pt,-30pt)$) -- ($(huvudmodul.south) + (40pt,-30pt)$) -- ($(huvudmodul.south) + (40pt,0)$);
		\draw[thick, ->] ($(styrmodul.south) + (+40pt,0)$) -- ($(styrmodul.south) + (40pt,-40pt)$) -- node [midway,below,minimum height=20pt, draw=none] {Avbrott}($(huvudmodul.south) + (30pt,-40pt)$) -- ($(huvudmodul.south) + (30pt,0)$);
	\end{tikzpicture}
\end{document}}
	\caption{Intermodulär kommunikation \label{communication}}
\end{figure}

Det finns två olika körlägen för roboten, ett autonomt och ett manuellt läge. När det autonoma läget är aktiverat tas beslut utifrån en avsökningsalgoritm som i grunden sker genom högerföljning och där beslutet om vilken kartmodul som ska besökas härnäst baseras på datan som skickats från sensormodulen. När beslutet är taget skickas motsvarande kommando till styrmodulen, som sedan utför önskad operation. Nytt beslut tas när styrmodulen anser sig vara färdig med nuvarande kommando och begär ett nytt genom avbrott. Hela kommunikationsflödet i autonomt läge finns beskrivet i figur \ref{autonomousMode}. I manuellt läge skickas styrkommandon från datormodulen, via Bluetooth\textsuperscript{\circledR}, till huvudmodulen som sen vidarebefodrar dessa till styrmodulen. Den informationen som skickas är begränsad till när en knapp trycks ned och när en knapp släpps, och i så fall vilken. Flödet finns beskrivet i figur \ref{manualMode}. Rent hårdvarumässigt styrs detta med hjälp av en switch kopplad till huvudmodulen.

Beräkningen av kortaste vägen till målet sker med utgångspunkt från Dijkstras algoritm. När målet är funnet numreras varje nod med en siffra som representerar ett avstånd till målet. Kortaste väg från start till mål fås då genom att, för varje nod välja den närliggande nod med lägst siffra. För att inte utforska mer än nödvändigt används en pessimistisk skattning av avståndet som motsvarar hur många steg från start som målet är om roboten följt vägen och undvikt återvändsgränder. Detta avstånd jämförs sedan med summan av hur köravståndet från målet till aktuell nod samt manhattanavståndet från start till aktuell nod. Är det uträknade avståndet lika eller större än skattningen är det inte nödvändigt att utforska mer åt det hållet.

\begin{figure}[htbp]
\centering
\noindent\resizebox{0.9\linewidth}{!}{
	\documentclass[border=10px]{standalone}
\usepackage{tikz}
\usetikzlibrary{patterns}
\usetikzlibrary{shapes.geometric}
\usetikzlibrary{shapes.arrows}
\usepackage{amssymb}
\usetikzlibrary{calc}
\usepackage{verbatim}

\pagestyle{empty}
\begin{document}

\tikzstyle{decision} = [diamond, draw,
    text width=5em, text badly centered, node distance=3cm, inner sep=0pt]
    
\tikzstyle{block} = [rectangle, draw,
    text width=7em, text centered, rounded corners, minimum height=4em]
	
\begin{tikzpicture}[scale=1]

%http://www.texample.net/tikz/examples/simple-flow-chart/

\node[block](start){Start};
\node[block, below of = start, node distance = 3cm](nyModul){Välj ny modul att utforska};
\node[block, right of = nyModul, node distance = 5cm](styrning){Skicka styrkommando till styrmodulen};
%\node[block, right of = styrning, node distance = 4.5cm](sensor){Meddela sensormodulen vilken sensordata som önskas till regleringen};
\node[decision, aspect=1.5,right of = styrning, node distance = 4.5cm](regleringKlar){Är regleringen klar?};
\node[block, right of=regleringKlar, node distance = 4.5cm] (sensorData) {Skicka sensordata till styrmodul};

\draw[->](start) -- (nyModul);
\draw[->](nyModul.east) -- (styrning);
\draw[->](styrning) -- (regleringKlar);
%\draw[->](sensor) -- (regleringKlar);
\draw[->](regleringKlar.east) -- node[above]{nej} (sensorData);
\draw[->](sensorData.north) -| ++(0,1.5) node(lowerright){} -| (regleringKlar.north);
\draw[->](regleringKlar.south) -| ++(0,-1.5) node[above right]{ja} -| (nyModul.south);

	\end{tikzpicture}
	
\end{document}
}
	\cprotect\caption{Flödesschema som beskriver förloppet vid autonom styrning \label{autonomousMode}}	
\end{figure}

\begin{figure}[htbp]
\centering
\noindent\resizebox{0.7\linewidth}{!}{
	\documentclass[border=10px]{standalone}
\usepackage{tikz}
\usetikzlibrary{patterns}
\usetikzlibrary{shapes.geometric}
\usetikzlibrary{shapes.arrows}
\usepackage{amssymb}
\usetikzlibrary{calc}
\usepackage{verbatim}

\pagestyle{empty}
\begin{document}

\tikzstyle{decision} = [diamond, draw,
    text width=4em, text badly centered, node distance=3cm, inner sep=0pt]
    
\tikzstyle{block} = [rectangle, draw,
    text width=5em, text centered, rounded corners, minimum height=4em]
	
\begin{tikzpicture}[scale=1]

%http://www.texample.net/tikz/examples/simple-flow-chart/

\node[block](start){Start};
%\node[decision, aspect=2, text width = 8em, below of = start, node distance = 3cm](kommando){Matar användare in styrkommando på tangentbordet?};
\node[block, right of = start, node distance = 4cm, text width = 7em](blåtand){Styrkommando från datormodulen tas emot};
\node[block, right of = blåtand, node distance = 4.5cm, text width = 7em](toStart){Styrkommando skickas från huvudmodul till styrmodul};
%\node[decision, aspect=2, text width = 5em, right of = toStart, node distance = 4.5cm](isDone){Har knappen släppts upp?};
%\node[block, below of = isDone,node distance = 8em](stop){Skicka stoppsignal till styrmodulen};

\draw[->](start) --  (blåtand);
\draw[->](blåtand) -- (toStart);
%\draw[->](toStart) -- (isDone);
%\draw[->](isDone) -- node[near start, right]{ja}(stop);
%\draw[->](isDone.east) -| node[near start, below]{nej} ++(0.7,2) -| (isDone.north);

	\end{tikzpicture}
	
\end{document}}
	\cprotect\caption{Flödesschema som beskriver förloppet vid manuell styrning \label{manualMode}}	
\end{figure}

\pagebreak

\subsection{Sensormodulen}
Robotens sensormodul har som uppgift att läsa av sensorerna och kommunicera datan till huvudmodulen. De sensorer som används är följande, NAMN PÅ SENSORER
\begin{description}
	\item[Avstånd] \hfill \\
	4 x IR-sensor \verb+GP2D120+ (4 cm till 30 cm) \\
	1 x Laser-sensor \verb+LIDAR-Lite v2+ (0 till 40 m) \\
	\item[Vinkelhastighet] \hfill \\
	1 x Gyro/accelerometer \verb+MPU-6500+ 
	\item[Identifierare av nödställd] \hfill \\
	1 x IR-detektor \verb+IRM-8601-S+
\end{description}
och är placerade enligt figur \ref{sensors}. 

\begin{figure}[htbp]
\centering
\noindent\resizebox{.8\textwidth}{!}{
	\documentclass[border=10px]{standalone}
\usepackage{tikz}
\usetikzlibrary{patterns}
\usetikzlibrary{shapes.arrows}
\usepackage{amssymb}
\usetikzlibrary{calc}
\usepackage{verbatim}
\begin{document}
	
\begin{tikzpicture}[scale=1,rotate=90]
		
	%Base
	\draw[thick, draw=black, fill=gray!10] (0,0) rectangle (6,10);

	%Wheels
	\draw[thick, pattern=north west lines, pattern color=black] (-.5,1) 		rectangle (0,2.5);
	\draw[thick, pattern=north west lines, pattern color=black] (-.5,7.5) 	rectangle (0,9);
	\draw[thick, pattern=north west lines, pattern color=black] (6,1) 		rectangle (6.5,2.5);
	\draw[thick, pattern=north west lines, pattern color=black] (6,7.5) 		rectangle (6.5,9);
	
	%Sensors
	\draw[thick, draw=black, fill=white] (-.25,.25) 		rectangle (.5,.75);
	\draw[thick, draw=black, fill=white] (-.25,9.25) 	rectangle (.5,9.75);
	\draw[thick, draw=black, fill=white] (5.5,.25) 		rectangle (6.25,.75);
	\draw[thick, draw=black, fill=white] (5.5,9.25) 		rectangle (6.25,9.75);
	\draw[thick, draw=black, fill=white] (2,10.25) 		rectangle (4,9.5);
	\draw[thick, draw=black, fill=white] (2.5,4) 		rectangle (3.5,6);
	
	%Arrows and text
	\draw[thick, ->]  (3,11) node[left, align=center] {\verb+LIDAR-Lite v2+ \\ + detektor av nödställd} -- (3,10.25);
	\draw[thick, <->] (0.5,0.5)  --  (5.5,0.5) node[left=-14pt,midway, fill=gray!10] {\verb+GP2D120+};
	\draw[thick, <->] (0.5,9.5) -- (3,9) node[right=-14pt,fill=gray!10] {\verb+GP2D120+} -- (5.5,9.5);
	\draw[thick, ->] (4.5,5) node[above] {\verb+MPU-6500+} -- (3.5,5);
	\end{tikzpicture}
	
\end{document}	}
	\caption{Sensorplacering \label{sensors}}
\end{figure}

\subsection{Styrmodulen}
Styrmodulen är den del av roboten som tar emot och verkställer styrkommandon och ser till att roboten tar sig fram på önskat sätt. Det som ska kunna hanteras är styrning av chassits motorer för att få roboten att röra på sig och att styra gripklon.

\subsection{Datormodulen}

\section{Intermodulär kommunikation}

\section{Slutsatser}
\textit{Vilka förbättringar skulle kunna göras?}

\pagebreak
\addcontentsline{toc}{section}{Referenser}
\bibliographystyle{ieeetr}
\bibliography{references}

\pagebreak


\appendix

\end{flushleft}

\end{document}
