\documentclass[11pt]{article}

\usepackage{extras} % Se extras.sty

\begin{document}
\begin{titlepage}
\begin{center}

{\Large\bfseries TSEA56 - Kandidatprojekt i elektronik \\ LIPS Teknisk Dokumentation}

\vspace{5em}

Version 1.0

\vspace{5em}
Grupp 4 \\
\begin{tabular}{rl}
Hynén Ulfsjöö, Olle&\verb+ollul666+
\\
Wasteson, Emil&\verb+emiwa068+
\\
Tronje, Elena&\verb+eletr654+
\\
Gustafsson, Lovisa&\verb+lovgu777+
\\
Inge, Zimon&\verb+zimin415+
\\
Strömberg, Isak&\verb+isast763+
\\
\end{tabular}

\vspace{5em}
\today

\vspace{16em}
Status
\begin{longtable}{|l|l|l|} \hline

Granskad & - & - \\ \hline
Godkänd & - & - \\ \hline
 
\end{longtable}

\end{center}
\end{titlepage}

\pagebreak
\begin{center}

\section*{PROJEKTIDENTITET}
2016/VT, Undsättningsrobot Gr. 4

Linköpings tekniska högskola, ISY
\vspace{5em}
%\begin{center}

\begin{tabular}{|l|l|l|l|} \hline
\textbf{Namn} & \textbf{Ansvar} & \textbf{Telefon} & \textbf{E-post}  \\ \hline 
Isak Strömberg (IS) & Projektledare & 073-980 38 50 & isast763@student.liu.se \\ \hline
Olle Hynén Ulfsjöö (OHU)& Dokumentansvarig & 070-072 91 84 & ollul666@student.liu.se \\ \hline
Emil Wasteson (EW) & Hårdvaruansvarig & 076-836 61 66 & emiwa068@student.liu.se \\ \hline
Elena Tronje (ET) & Mjukvaruansvarig & 072-276 92 93 & eletr654@student.liu.se \\ \hline
Zimon Inge (ZI)& Testansvarig & 070-171 35 18 & zimin415@student.liu.se \\ \hline
Lovisa Gustafsson (LG) & Leveransansvarig & 070-210 32 53 & lovgu777@student.liu.se \\ \hline
\end{tabular}

%\end{center}

E-postlista för hela gruppen: isast763@student.liu.se

\vspace{5em}
Kund: ISY, Linköpings universitet 
tel: 013-28 10 00, fax: 013-13 92 82 \\
Kontaktperson hos kund: Mattias Krysander \\
tel: 013-28 21 98, e-post: matkr@isy.liu.se \\

\vspace{5em}
Kursansvarig:  Tomas Svensson\\
tel: 013-28 13 68, e-post: tomass@isy.liu.se \\
Handledare: Peter Johansson \\
tel: 013-28 13 45, e-post: peter.a.johansson@liu.se
\end{center}
\pagebreak

\tableofcontents

\pagebreak

\section*{Dokumenthistorik}
\begin{table}[h]
\begin{tabular}{|l|l|l|l|l|} \hline

\textbf{Version} & \textbf{Datum} & \textbf{Utförda förändringar} & \textbf{Utförda av} & \textbf{Granskad} \\ \hline
0.1 & - &  Första utkastet & Grupp 4 & - \\ \hline
\end{tabular}
\end{table}

\pagebreak
\pagenumbering{arabic}

\begin{flushleft}
\section{Inledning}
Det genomförda projektet har gått ut på att konstruera en undsättningsrobot. Detta dokument syftar till att användas som underlag till personer som vill bygga en liknande robot. Innehåller består av detaljerade beskrivningar av systemets olika delar och dess mjuk- och hårdvarukomponenter samt implementation.

\section{Produkten}
Produkten som har konstruerats är en undsättningsrobot bestående av fyra moduler som tillsammans med sensorer och en gripklo är integrerade på ett gemensamt chassi, enligt figur \ref{overview}.  ÄNDRA TILL 1284P OCH EV GP2D120

\begin{figure}[!htbp]
\centering
\noindent\resizebox{\linewidth}{!}{
	\documentclass{minimal}
\usepackage{tikz}
%\usetikzlibrary{calc,trees,positioning,arrows,chains,shapes.geometric,decorations.pathreplacing,decorations.pathmorphing,shapes,matrix,shapes.symbols}
\usetikzlibrary{positioning}
\usetikzlibrary{shapes}

\begin{document}
\begin{center}
\begin{tikzpicture}[scale=1]
\tikzset{every node/.style={inner sep=10pt, minimum width=3 cm}}
%\draw[help lines,step=5mm,gray!20] (-5,-10) grid (5,0);
\node[draw, fill=white] (Huvudmodul)  {\textbf{Huvudmodul}};
\node[draw,below left= of Huvudmodul] (Sensormodul) {\textbf{Sensormodul}};
\node[draw,below right = of Huvudmodul] (Styrenhet) {\textbf{Styrenhet}};
\node[draw, above = of Huvudmodul] (Datormodul) {\textbf{Datormodul}};
\node[ellipse,draw, right = of Datormodul] (Användare) {\textbf{Användare}};

\draw[->] (Huvudmodul) [out=300, in= 90] to (Styrenhet);
\draw[->] (Sensormodul) [out=90, in=240] to (Huvudmodul);
\draw[<->] (Huvudmodul) to (Datormodul);
\draw[<->] (Datormodul) to (Användare);
\end{tikzpicture}
\end{center}

\vspace{10em}

\begin{center}
\begin{tikzpicture}[scale=1]
\tikzset{every node/.style={inner sep=10pt, minimum width=3 cm}}
%\draw[help lines,step=5mm,gray!20] (-5,-10) grid (5,0);

\node[draw] (Sensormodul) {\textbf{Sensormodul}};
\node[above right = of Sensormodul,minimum width=0,inner sep=2pt] (Huvudmodul) {Huvudmodul};

\node[below left = of Sensormodul, minimum width = 0, inner sep = 2pt] (Sensor) {Sensorer};


\draw[->] (Sensor) [out=0,in=270] to node [sloped, midway, below] {spänningsnivåer} (Sensormodul);
\draw[->] (Sensormodul) [out=90,in=180] to node [sloped,midway, above] {enheter}  (Huvudmodul);

\end{tikzpicture}
\end{center}

\vspace{10em}

\begin{center}
\begin{tikzpicture}[scale=1]
\tikzset{every node/.style={inner sep=10pt, minimum width=3 cm}}
%\draw[help lines,step=5mm,gray!20] (-5,-10) grid (5,0);

\node[draw] (Styrmodul) {\textbf{Styrmodul}};
\node[above left = of Styrmodul,minimum width=0,inner sep=2pt] (Huvudmodul) {Huvudmodul};

\node[below right = of Styrmodul, minimum width = 0, inner sep = 2pt] (Motorer) {Motorer};
%\node[below left = of Styrmodul, minimum width = 0, inner sep= 2pt] (Gripklo) {Gripklo}


\draw[->] (Styrmodul) [out=270,in=180] to node [sloped, midway, below] {spänningsnivåer} (Motorer);
\draw[<-] (Sensormodul) [out=90,in=0] to node [sloped,midway, above] {kommandon}  (Huvudmodul);

\end{tikzpicture}
\end{center}

\vspace{10em}

\begin{center}
\begin{tikzpicture}[scale=1]
\tikzset{every node/.style={inner sep=10pt, minimum width=3 cm}}
%\draw[help lines,step=5mm,gray!20] (-5,-10) grid (5,0);

\node[draw] (Datormodul) {\textbf{Datormodul}};
\node[below = 10 em of Datormodul,minimum width=0,inner sep=2pt] (Huvudmodul) {Huvudmodul};
\node[right = of Datormodul, ellipse, draw] (Användare) {Användare};

%\node[below right = of Styrmodul, minimum width = 0, inner sep = 2pt] (Motorer) {Motorer};
%\node[below left = of Styrmodul, minimum width = 0, inner sep= 2pt] (Gripklo) {Gripklo}


\draw[->] (Datormodul) [out=30,in=150] to node [midway, above] {karta} (Användare);
\draw[<-] (Datormodul) [out=-30,in=210] to node [midway, below] {kommandon}  (Användare);

\draw[->] (Datormodul) [out=300,in=60] to node [near end, right] {kommandon} (Huvudmodul);
\draw[<-] (Datormodul) [out=240,in=120] to node [near end, left] {sensordata} (Huvudmodul);

\end{tikzpicture}
\end{center}


\end{document}}
	\caption{Det totala systemet \label{overview}}	
\end{figure}

Med hjälp av sensorerna kan roboten navigera i ett laryrintsystem, vilket den gör ända till den detekterar en nödställd. Då ser den till att finna den snabbaste/kortaste vägen ut och hämtar en förnödenhet med gripklon som den sedan åker och lämnar till den nödställde. 

\section{Teori}

\section{Systemet}

\subsection{Protokoll}


\section{Modulerna}
I detta avsnitt följer en detaljerad beskrivning av systemets ingående moduler. 

\subsection{Huvudmodulen}
Huvudmodulen har som uppgift att ta emot och förmedla information mellan de andra modulera samt att göra alla tyngre beräkningar. Kommunikationen med den externa datormodulen sker genom Bluetooth\textsuperscript{\circledR} och genom en tvåtrådad I\textsuperscript{2}C-buss med de interna modulerna, vilket visas i figur \ref{communication}.

\begin{figure}[htbp]
\noindent\resizebox{.97\textwidth}{!}{
	\documentclass[border=20pt]{standalone}
\usepackage{tikz}
\usetikzlibrary{positioning}
\usetikzlibrary{calc}
\usetikzlibrary{decorations.pathmorphing}
\usepackage{amssymb}
\usetikzlibrary{shapes,arrows}

\begin{document}
	\begin{tikzpicture}[scale=1]
		
		\tikzset{every node/.style={thick, draw=black, align=center, minimum height=40pt, text width=100pt, minimum width=100pt}}
		\node(datormodul) {Datormodul};
		\node[right=10pt of datormodul,minimum height=20pt, minimum width=10pt,text width=10pt] (bt1) {\includegraphics{bluetooth}};
		
		\node[right=40pt of bt1,minimum height=20pt, minimum width=10pt,text width=10pt] (bt2) {\includegraphics{bluetooth}};
		
		\node[right=10pt of bt2] 			(huvudmodul) 	{Huvudmodul};
		\node[below=-10pt of huvudmodul,draw=none] (master) {\textit{master}};
		\node[right=10pt of huvudmodul] 		(sensormodul) 	{Sensormodul};
		\node[below=-10pt of sensormodul,draw=none] (slave1) {\textit{slav}};
		\node[right=10pt of sensormodul] 	(styrmodul) 		{Styrmodul};
		\node[below=-10pt of styrmodul,draw=none] (slave2) {\textit{slav}};
		
		\coordinate (sclStart) 	at ($(huvudmodul.north west) + (0,20pt)$);
		\coordinate (sclEnd)		at ($(styrmodul.north east)  + (0,20pt)$);
		
		\coordinate (sdaStart)  at ($(sclStart) + (0,20pt)$);
		\coordinate (sdaEnd)		at ($(sclEnd)	+ (0,20pt)$);
		
		\coordinate (vddStart)  at ($(sdaStart) + (0,40pt)$);
		\coordinate (vddEnd)		at ($(sdaEnd)	+ (0,40pt)$);
		
		\draw[thick] (sclStart) -- (sclEnd) node [right,draw=none,text width=0,minimum width=0] {SCL};
		\draw[thick] (sdaStart) -- (sdaEnd) node [right,draw=none,text width=0,minimum width=0] {SDA};
		\draw[thick] (vddStart) -- (vddEnd) node [right,draw=none,text width=0,minimum width=0] {$V_{dd}$};
		
		\draw[thick] (datormodul.east) -- (bt1.west);
		
		\draw[thick, ->,line join=round,decorate, decoration={
    												snake,
    												segment length=5,
    												amplitude=1,
    												post=lineto,
    												post length=1pt}] 
    		($(bt1.east) + (5pt,5pt)$) -- ($(bt2.west) + (-5pt,5pt)$);
    		
    	\draw[thick, ->,line join=round,decorate, decoration={
    												snake,
    												segment length=5,
    												amplitude=1,
    												post=lineto,
    												post length=1pt}] 
    		 ($(bt2.west) + (-5pt,-5pt)$) -- ($(bt1.east) + (5pt,-5pt)$);
    		 
    	\draw[thick] (bt2.east) -- (huvudmodul.west);
    	
    	\draw[thick,fill=black] ($(huvudmodul.north) + (-10pt,0)$) -- ($(huvudmodul.north) + (-10pt,20pt)$) circle [radius=2pt];
    	\draw[thick,fill=black] ($(huvudmodul.north) + (+10pt,0)$) -- ($(huvudmodul.north) + (+10pt,40pt)$) circle [radius=2pt];
    	
    	\draw[thick,fill=black] ($(sensormodul.north) + (-10pt,0)$) -- ($(sensormodul.north) + (-10pt,20pt)$) circle [radius=2pt];
    	\draw[thick,fill=black] ($(sensormodul.north) + (+10pt,0)$) -- ($(sensormodul.north) + (+10pt,40pt)$) circle [radius=2pt];
    	
    	\draw[thick,fill=black] ($(styrmodul.north) + (-10pt,0)$) -- ($(styrmodul.north) + (-10pt,20pt)$) circle [radius=2pt];
    	\draw[thick,fill=black] ($(styrmodul.north) + (+10pt,0)$) -- ($(styrmodul.north) + (+10pt,40pt)$) circle [radius=2pt];
    	
    	\draw[thick,fill=black] ($(sensormodul.north east) + (-5pt,20pt)$) circle [radius=2pt] -- ($(sensormodul.north east) + (-5pt,80pt)$) circle [radius=2pt];
    	\draw[thick,fill=black] ($(styrmodul.north west) + (5pt,40pt)$) circle [radius=2pt] -- ($(styrmodul.north west) + (5pt,80pt)$) circle [radius=2pt];
    	\draw[thick,draw=black,fill=white] ($(styrmodul.north west) + (2pt,50pt)$) rectangle ($(styrmodul.north west) + (8pt,70pt)$);
		\draw[thick,draw=black,fill=white] ($(sensormodul.north east) + (-8pt,50pt)$) rectangle ($(sensormodul.north east) + (-2pt,70pt)$) node [below left=-10pt and 20pt,draw=none,minimum width=0, text width = 0pt] {$R_p$};
		
		\draw[thick, ->] ($(sensormodul.south) + (+40pt,0)$) -- ($(sensormodul.south) + (40pt,-30pt)$) -- ($(huvudmodul.south) + (40pt,-30pt)$) -- ($(huvudmodul.south) + (40pt,0)$);
		\draw[thick, ->] ($(styrmodul.south) + (+40pt,0)$) -- ($(styrmodul.south) + (40pt,-40pt)$) -- node [midway,below,minimum height=20pt, draw=none] {Avbrott}($(huvudmodul.south) + (30pt,-40pt)$) -- ($(huvudmodul.south) + (30pt,0)$);
	\end{tikzpicture}
\end{document}}
	\caption{Intermodulär kommunikation \label{communication}}
\end{figure}

\pagebreak

\subsection{Sensormodulen}
Robotens sensormodul har som uppgift att läsa av den data som behövs för att reglera, identifiera den nödställda och att kommunicera datan till huvudmodulen. De sensorer som används är följande, NAMN PÅ SENSORER
\begin{description}
	\item[Avstånd] \hfill \\
	4 x IR-sensor \verb+GP2D120+ (4 cm till 30 cm) \\
	1 x Laser-sensor \verb+LIDAR-Lite v2+ (0 till 40 m) \\
	\item[Vinkelhastighet] \hfill \\
	1 x Gyro/accelerometer \verb+MPU-6500+ 
	\item[Identifierare av nödställd] \hfill \\
	1 x IR-detektor \verb+IRM-8601-S+
\end{description}
och är placerade enligt figur \ref{sensors}. 

\begin{figure}[htbp]
\centering
\noindent\resizebox{.8\textwidth}{!}{
	\documentclass[border=10px]{standalone}
\usepackage{tikz}
\usetikzlibrary{patterns}
\usetikzlibrary{shapes.arrows}
\usepackage{amssymb}
\usetikzlibrary{calc}
\usepackage{verbatim}
\begin{document}
	
\begin{tikzpicture}[scale=1,rotate=90]
		
	%Base
	\draw[thick, draw=black, fill=gray!10] (0,0) rectangle (6,10);

	%Wheels
	\draw[thick, pattern=north west lines, pattern color=black] (-.5,1) 		rectangle (0,2.5);
	\draw[thick, pattern=north west lines, pattern color=black] (-.5,7.5) 	rectangle (0,9);
	\draw[thick, pattern=north west lines, pattern color=black] (6,1) 		rectangle (6.5,2.5);
	\draw[thick, pattern=north west lines, pattern color=black] (6,7.5) 		rectangle (6.5,9);
	
	%Sensors
	\draw[thick, draw=black, fill=white] (-.25,.25) 		rectangle (.5,.75);
	\draw[thick, draw=black, fill=white] (-.25,9.25) 	rectangle (.5,9.75);
	\draw[thick, draw=black, fill=white] (5.5,.25) 		rectangle (6.25,.75);
	\draw[thick, draw=black, fill=white] (5.5,9.25) 		rectangle (6.25,9.75);
	\draw[thick, draw=black, fill=white] (2,10.25) 		rectangle (4,9.5);
	\draw[thick, draw=black, fill=white] (2.5,4) 		rectangle (3.5,6);
	
	%Arrows and text
	\draw[thick, ->]  (3,11) node[left, align=center] {\verb+LIDAR-Lite v2+ \\ + detektor av nödställd} -- (3,10.25);
	\draw[thick, <->] (0.5,0.5)  --  (5.5,0.5) node[left=-14pt,midway, fill=gray!10] {\verb+GP2D120+};
	\draw[thick, <->] (0.5,9.5) -- (3,9) node[right=-14pt,fill=gray!10] {\verb+GP2D120+} -- (5.5,9.5);
	\draw[thick, ->] (4.5,5) node[above] {\verb+MPU-6500+} -- (3.5,5);
	\end{tikzpicture}
	
\end{document}	}
	\caption{Sensorplacering \label{sensors}}
\end{figure}

\subsection{Styrmodulen}
Styrmodulen är den del av roboten som tar emot och verkställer styrkommandon och ser till att roboten tar sig fram på önskat sätt. Det som ska kunna hanteras är styrning av chassits motorer för att få roboten att röra på sig, rotera det eventuella lasersensortornet eller att styra gripklon.

\subsection{Datormodulen}


\section{Slutsatser}

\pagebreak
\addcontentsline{toc}{section}{Referenser}
\bibliographystyle{ieeetr}
\bibliography{references}

\pagebreak

\appendix

\end{flushleft}

\end{document}