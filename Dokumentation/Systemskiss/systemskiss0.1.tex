\documentclass[11pt]{article}

\usepackage{extras} % Se extras.sty
\usepackage{tikz}
\usepackage{parskip}
\usepackage{standalone}
%\usepackage{titlesec}
\usetikzlibrary{positioning}
\usetikzlibrary{shapes}
\usetikzlibrary{calc}
\usepackage{enumitem}
\def\arraystretch{1.5}
\graphicspath{ {images/} }
\setlength{\LTpost}{0pt}
\pagenumbering{roman}
\setlength{\LTpost}{0pt}%
\usepackage{graphicx}
\usetikzlibrary{patterns}
\usetikzlibrary{shapes.arrows}

%\usepackage[mode=buildnew]{standalone}

%\titlespacing*{\subsection}{0pt}{1.1\baselineskip}{\baselineskip}


\begin{document}
\begin{titlepage}
\begin{center}

{\Large\bfseries TSEA56 - Kandidatprojekt i elektronik \\ LIPS Systemskiss}

\vspace{5em}

Version 0.1

\vspace{5em}
%
Grupp 4 \\
\begin{tabular}{rl}
Hynén Ulfsjöö, Olle&\verb+ollul666+
\\
Wasteson, Emil&\verb+emiwa068+
\\
Tronje, Elena&\verb+eletr654+
\\
Gustafsson, Lovisa&\verb+lovgu777+
\\
Inge, Zimon&\verb+zimin415+
\\
Strömberg, Isak&\verb+isast763+
\\
\end{tabular}

\vspace{5em}
\today

\vspace{16em}
Status
\begin{longtable}{|l|l|l|} \hline

Granskad & - & - \\ \hline
Godkänd & - & - \\ \hline
 
\end{longtable}

\end{center}
\end{titlepage}

\pagebreak
\begin{center}

\section*{PROJEKTIDENTITET}
2016/VT, Undsättningsrobot Gr. 4

Linköpings tekniska högskola, ISY
\vspace{5em}
\begin{center}

\begin{tabular}{|l|l|l|l|} \hline
\textbf{Namn} & \textbf{Ansvar} & \textbf{Telefon} & \textbf{E-post}  \\ \hline 
Isak Strömberg (IS) & Projektledare & 073-980 38 50 & isast763@student.liu.se \\ \hline
Olle Hynén Ulfsjöö (OHU)& Dokumentansvarig & 070-072 91 84 & ollul666@student.liu.se \\ \hline
Emil Wasteson (EW) & Hårdvaruansvarig & 076-836 61 66 & emiwa068@student.liu.se \\ \hline
Elena Tronje (ET) & Mjukvaruansvarig & 072-276 92 93 & eletr654@student.liu.se \\ \hline
Zimon Inge (ZI)& Testansvarig & 070-171 35 18 & zimin415@student.liu.se \\ \hline
Lovisa Gustafsson (LG) & Leveransansvarig & 070-210 32 53 & lovgu777@student.liu.se \\ \hline
\end{tabular}

\end{center}

E-postlista för hela gruppen: isast763@student.liu.se

\vspace{5em}
Kund: ISY, Linköpings universitet \\
tel: 013-28 10 00, fax: 013-13 92 82 \\
Kontaktperson hos kund: Mattias Krysander \\
tel: 013-28 21 98, e-post: matkr@isy.liu.se \\

\vspace{5em}
Kursansvarig:  Tomas Svensson\\
tel: 013-28 13 68, e-post: tomass@isy.liu.se \\
Handledare: Peter Johansson \\
tel: 013-28 13 45, epost: peter.a.johansson@liu.se
\end{center}
\pagebreak

\tableofcontents

\pagebreak

\section*{Dokumenthistorik}
\begin{table}[h]
\begin{tabular}{|l|l|l|l|l|} \hline

\textbf{Version} & \textbf{Datum} & \textbf{Utförda förändringar} & \textbf{Utförda av} & \textbf{Granskad} \\ \hline
0.1 & FYLL I &  Första utkastet & Grupp 4 & - \\ \hline
\end{tabular}
\end{table}

\pagebreak
\pagenumbering{arabic}

\begin{flushleft}

\section{Inledning}
Denna systemskiss utgör en del av förarbetet i kandidatprojektet TSEA56. På uppdrag av beställaren ska systemskissen ge en detaljerad beskrivning ar produkten och dess delsystem. Dokumentet inleds med en grov beskrivning av produkten, därefter följer en mer detaljerad beskrivning på modulnivå.
\subsection{Syfte och mål}
Syftet med projektet är att bygga en undsättningsrobot på prototypnivå. Roboten ska autonomnt kunna utforska en \textit{simulerad} grotta och finna en nödställd. Parallellt med sökande ska roboten kunna kommunicera med en datormodul där en karta av grottan successivt ritas upp. När den nödställde är funnen ska en optimerande algoritm beräkna den kortaste vägen dit. Denna väg ska sedan användas när roboten ska förse ned nödställde med någon förnödenhet. Dessa krav finns specificerade i kravspecifikationen.\cite{krav}

\pagebreak
\section{Översikt av systemet}

\begin{figure}[htbp]
\centering
\begin{tikzpicture}[scale=1.2]
		
		\draw[thick] 	(0,0) -- (0,3);
		\draw[thick] 	(1,7) -- (6,7) -- (6,2);
					
		\draw[thick,pattern=north west lines, pattern color=black]	 	(4,0) -- (4,2) -- (6,2) -- (6,0) -- (4,0);
		
		\draw[thick,pattern=north west lines, pattern color=black]
						(0,3) -- (3,3) -- (3,4) -- (1,4)
					--	(1,7) -- (0,7) -- (0,3);
		
		\draw[thick,pattern=north west lines, pattern color=black]
						(2,5) -- (3,5) -- (3,6) -- (2,6)
					--	(2,5);
					
		\draw[thick,pattern=north west lines, pattern color=black]		(4,3) -- (5,3) -- (5,6) -- (4,6)
					--	(4,3);
					
		\draw[thick,pattern=north west lines, pattern color=black]
				(4,0) -- (1,0) -- (1,2) -- (3,2) -- (3,0);
				
		\draw[thick,->] (0.5,0) -- (0.5,0.5);
		
		\draw[thick] (4.2,2.2) rectangle (5,2.8) node[pos=.5] {\tiny Robot};
		\draw[thick] (1.2,6.2) rectangle (1.8,6.8);
		\draw[thick,<-] (1.8,6.5) -- (2.3,6.5) node[right] {\tiny Nödställd};
		\draw[thick, <-,overlay] (5.1,2.5) [out=0,in=-45] to (7,4) node [right=.5em] {\tiny Bluetooth};
		\draw[thick,->,overlay] (7,4) [out=125,in=180] to (9,5) ;
		\node[]  at (10,5) {\includegraphics[scale=0.8]{laptop}};
		\node[overlay] at (10,4) {Datormodul};
	\end{tikzpicture}
	\caption{Översikt av systemet \label{system}}	
\end{figure}


\subsection{Beskrivning av systemet}
text
\subsection{Delsystem}
text
\subsection{Kommunikation mellan delsystem}
text

\pagebreak
\section{Delsystem 1 - Huvudmodul}
text
\subsection{LCD-display}
text
\subsection{Brytare}
% Typ auto/manuell styrning

\pagebreak
\section{Delsystem 2 - Sensormodul}
text
\subsection{Sensorer}
text

\pagebreak
\section{Delsystem 3 - Styrmodul}
text
\subsection{Motorer}
text
\subsection{Gripklo}
text

\pagebreak
\section{Delsystem 4 - Datormodul}
text
\subsection{Mjukvara}
text
\subsection{Gränssnitt}
text

\setcounter{secnumdepth}{0}
\pagebreak
\begin{thebibliography}{9}

\bibitem{krav}
  Grupp 4,
  \emph{Kravspecifikation 1.0},
  2016-02-03

\end{thebibliography}

\setcounter{secnumdepth}{2}

\end{flushleft}

\end{document}