\documentclass[11pt]{article}

\usepackage{extras} % Se extras.sty

\begin{document}
\begin{titlepage}
\begin{center}

{\Large\bfseries TSEA56 - Kandidatprojekt i elektronik \\ LIPS Systemskiss}

\vspace{5em}

Version 0.1

\vspace{5em}
Grupp 4 \\
\begin{tabular}{rl}
Hynén Ulfsjöö, Olle&\verb+ollul666+
\\
Wasteson, Emil&\verb+emiwa068+
\\
Tronje, Elena&\verb+eletr654+
\\
Gustafsson, Lovisa&\verb+lovgu777+
\\
Inge, Zimon&\verb+zimin415+
\\
Strömberg, Isak&\verb+isast763+
\\
\end{tabular}

\vspace{5em}
\today

\vspace{16em}
Status
\begin{longtable}{|l|l|l|} \hline

Granskad & - & - \\ \hline
Godkänd & - & - \\ \hline
 
\end{longtable}

\end{center}
\end{titlepage}

\pagebreak
\begin{center}

\section*{PROJEKTIDENTITET}
2016/VT, Undsättningsrobot Gr. 4

Linköpings tekniska högskola, ISY
\vspace{5em}
\begin{center}

\begin{tabular}{|l|l|l|l|} \hline
\textbf{Namn} & \textbf{Ansvar} & \textbf{Telefon} & \textbf{E-post}  \\ \hline 
Isak Strömberg (IS) & Projektledare & 073-980 38 50 & isast763@student.liu.se \\ \hline
Olle Hynén Ulfsjöö (OHU)& Dokumentansvarig & 070-072 91 84 & ollul666@student.liu.se \\ \hline
Emil Wasteson (EW) & Hårdvaruansvarig & 076-836 61 66 & emiwa068@student.liu.se \\ \hline
Elena Tronje (ET) & Mjukvaruansvarig & 072-276 92 93 & eletr654@student.liu.se \\ \hline
Zimon Inge (ZI)& Testansvarig & 070-171 35 18 & zimin415@student.liu.se \\ \hline
Lovisa Gustafsson (LG) & Leveransansvarig & 070-210 32 53 & lovgu777@student.liu.se \\ \hline
\end{tabular}

\end{center}

E-postlista för hela gruppen: isast763@student.liu.se

\vspace{5em}
Kund: ISY, Linköpings universitet \\
tel: 013-28 10 00, fax: 013-13 92 82 \\
Kontaktperson hos kund: Mattias Krysander \\
tel: 013-28 21 98, e-post: matkr@isy.liu.se \\

\vspace{5em}
Kursansvarig:  Tomas Svensson\\
tel: 013-28 13 68, e-post: tomass@isy.liu.se \\
Handledare: Peter Johansson \\
tel: 013-28 13 45, epost: peter.a.johansson@liu.se
\end{center}
\pagebreak

\tableofcontents

\pagebreak

\section*{Dokumenthistorik}
\begin{table}[h]
\begin{tabular}{|l|l|l|l|l|} \hline

\textbf{Version} & \textbf{Datum} & \textbf{Utförda förändringar} & \textbf{Utförda av} & \textbf{Granskad} \\ \hline
0.1 & FYLL I &  Första utkastet & Grupp 4 & - \\ \hline
\end{tabular}
\end{table}

\pagebreak
\pagenumbering{arabic}

\begin{flushleft}

\section{Inledning}
Denna systemskiss utgör en del av förarbetet i kandidatprojektet TSEA56. På uppdrag av beställaren ska systemskissen ge en detaljerad beskrivning ar produkten och dess delsystem. Dokumentet inleds med en grov beskrivning av produkten, därefter följer en mer detaljerad beskrivning på modulnivå.
\subsection{Syfte och mål}
Syftet med projektet är att bygga en undsättningsrobot på prototypnivå. Roboten ska autonomnt kunna utforska en \textit{simulerad} grotta och finna en nödställd. Parallellt med sökande ska roboten kunna kommunicera med en datormodul där en karta av grottan successivt ritas upp. När den nödställde är funnen ska en optimerande algoritm beräkna den kortaste vägen dit. Denna väg ska sedan användas när roboten ska förse den nödställde med någon förnödenhet. Dessa krav finns specificerade i kravspecifikationen.\cite{krav}

\pagebreak
\section{Översikt av systemet}
Roboten i sin miljö finns illustrerad i figur \ref{system}. Kommunikationen med datormodulen sker åt båda hållen och via Bluetooth\textsuperscript{\circledR}. Roboten ska dock även klara sitt uppdrag utan kommunikation med datormodulen. Det vill säga att kartläggning, styrning och optimering av kortaste väg sker lokalt på roboten. Banan är uppbyggd enligt banspecifikationen \cite{banspec} och uppdraget utförs enligt tävlingsreglerna \cite{tavling}.
\begin{figure}[htbp]
\centering
\noindent\resizebox{.8\linewidth}{!}{
	\documentclass{minimal}
\usepackage{tikz}
%\usetikzlibrary{calc,trees,positioning,arrows,chains,shapes.geometric,decorations.pathreplacing,decorations.pathmorphing,shapes,matrix,shapes.symbols}
\usetikzlibrary{positioning}
\usetikzlibrary{shapes}

\begin{document}
\begin{center}
\begin{tikzpicture}[scale=1]
\tikzset{every node/.style={inner sep=10pt, minimum width=3 cm}}
%\draw[help lines,step=5mm,gray!20] (-5,-10) grid (5,0);
\node[draw, fill=white] (Huvudmodul)  {\textbf{Huvudmodul}};
\node[draw,below left= of Huvudmodul] (Sensormodul) {\textbf{Sensormodul}};
\node[draw,below right = of Huvudmodul] (Styrenhet) {\textbf{Styrenhet}};
\node[draw, above = of Huvudmodul] (Datormodul) {\textbf{Datormodul}};
\node[ellipse,draw, right = of Datormodul] (Användare) {\textbf{Användare}};

\draw[->] (Huvudmodul) [out=300, in= 90] to (Styrenhet);
\draw[->] (Sensormodul) [out=90, in=240] to (Huvudmodul);
\draw[<->] (Huvudmodul) to (Datormodul);
\draw[<->] (Datormodul) to (Användare);
\end{tikzpicture}
\end{center}

\vspace{10em}

\begin{center}
\begin{tikzpicture}[scale=1]
\tikzset{every node/.style={inner sep=10pt, minimum width=3 cm}}
%\draw[help lines,step=5mm,gray!20] (-5,-10) grid (5,0);

\node[draw] (Sensormodul) {\textbf{Sensormodul}};
\node[above right = of Sensormodul,minimum width=0,inner sep=2pt] (Huvudmodul) {Huvudmodul};

\node[below left = of Sensormodul, minimum width = 0, inner sep = 2pt] (Sensor) {Sensorer};


\draw[->] (Sensor) [out=0,in=270] to node [sloped, midway, below] {spänningsnivåer} (Sensormodul);
\draw[->] (Sensormodul) [out=90,in=180] to node [sloped,midway, above] {enheter}  (Huvudmodul);

\end{tikzpicture}
\end{center}

\vspace{10em}

\begin{center}
\begin{tikzpicture}[scale=1]
\tikzset{every node/.style={inner sep=10pt, minimum width=3 cm}}
%\draw[help lines,step=5mm,gray!20] (-5,-10) grid (5,0);

\node[draw] (Styrmodul) {\textbf{Styrmodul}};
\node[above left = of Styrmodul,minimum width=0,inner sep=2pt] (Huvudmodul) {Huvudmodul};

\node[below right = of Styrmodul, minimum width = 0, inner sep = 2pt] (Motorer) {Motorer};
%\node[below left = of Styrmodul, minimum width = 0, inner sep= 2pt] (Gripklo) {Gripklo}


\draw[->] (Styrmodul) [out=270,in=180] to node [sloped, midway, below] {spänningsnivåer} (Motorer);
\draw[<-] (Sensormodul) [out=90,in=0] to node [sloped,midway, above] {kommandon}  (Huvudmodul);

\end{tikzpicture}
\end{center}

\vspace{10em}

\begin{center}
\begin{tikzpicture}[scale=1]
\tikzset{every node/.style={inner sep=10pt, minimum width=3 cm}}
%\draw[help lines,step=5mm,gray!20] (-5,-10) grid (5,0);

\node[draw] (Datormodul) {\textbf{Datormodul}};
\node[below = 10 em of Datormodul,minimum width=0,inner sep=2pt] (Huvudmodul) {Huvudmodul};
\node[right = of Datormodul, ellipse, draw] (Användare) {Användare};

%\node[below right = of Styrmodul, minimum width = 0, inner sep = 2pt] (Motorer) {Motorer};
%\node[below left = of Styrmodul, minimum width = 0, inner sep= 2pt] (Gripklo) {Gripklo}


\draw[->] (Datormodul) [out=30,in=150] to node [midway, above] {karta} (Användare);
\draw[<-] (Datormodul) [out=-30,in=210] to node [midway, below] {kommandon}  (Användare);

\draw[->] (Datormodul) [out=300,in=60] to node [near end, right] {kommandon} (Huvudmodul);
\draw[<-] (Datormodul) [out=240,in=120] to node [near end, left] {sensordata} (Huvudmodul);

\end{tikzpicture}
\end{center}


\end{document}}
	\caption{Översikt av systemet \label{system}}	
\end{figure}


\subsection{Beskrivning av systemet}
Roboten navigerar själva banan med hjälp av flertal sensorer, dessa finns specificerade i \mbox{kapitel \ref{sec:sensormodul}}. En regleringsmodell ser till att roboten färdas i mitten av korridorerna och kan ta svängar utan att stöta mot väggar. Under färden ska roboten autonomt kartlägga och finna den kortaste vägen mellan ingången och den nödställde. Ifall roboten även är uppkopplad mot datormodulen ska mjukvara på datorn successivt rita upp en karta och kunna presentera utvalda mätvärden i realtid. När den kortaste vägen är funnen ska roboten använda denna rutt för att förse den nödställde med en förnödenhet. Förnödenheten transporteras med hjälp av robotens gripklo.

\subsection{Delsystem}
Delsystemen och dess ingående komponenter finns beskrivna nedan.
\begin{description}
	\item[Delsystem 1 - Huvudmodul] \hfill \\
	Mikroprocessor \\
	Brytare för autonomt/manuellt-läge \\
	Bluetooth\textsuperscript{\circledR} sändare/mottagare
	\item[Delsystem 2 - Sensormodul] \hfill \\
	Mikroprocessor \\
	Sensorer
	\item[Delsystem 3 - Styrmodul] \hfill \\
	Mikroprocessor \\
	Motorer till hjul och gripklo \\
	LCD-display
	\item[Delsystem 4 - Datormodul] \hfill \\
	Dator \\
	GUI \\
	Bluetooth\textsuperscript{\circledR} sändare/mottagare
\end{description}

Ovanstående mikroprocessorer är av typ \verb+ATmega**+ där versionen bestäms i ett senare skede. Sensorerna består av avståndsmätare, vinkelhastighets-sensorer och identifierare av nödställd. 
\subsection{Kommunikation mellan delsystem}
Kommunikationen mellan mikroprocessorerna sker med hjälp av en I\textsuperscript{2}C-buss enligt figur \ref{communication}. Mellan huvudmodulen och datormodulen sker kommunikationen via Bluetooth\textsuperscript{\circledR}.

\begin{figure}[htbp]
\noindent\resizebox{.97\textwidth}{!}{
	\documentclass[border=20pt]{standalone}
\usepackage{tikz}
\usetikzlibrary{positioning}
\usetikzlibrary{calc}
\usetikzlibrary{decorations.pathmorphing}
\usepackage{amssymb}
\usetikzlibrary{shapes,arrows}

\begin{document}
	\begin{tikzpicture}[scale=1]
		
		\tikzset{every node/.style={thick, draw=black, align=center, minimum height=40pt, text width=100pt, minimum width=100pt}}
		\node(datormodul) {Datormodul};
		\node[right=10pt of datormodul,minimum height=20pt, minimum width=10pt,text width=10pt] (bt1) {\includegraphics{bluetooth}};
		
		\node[right=40pt of bt1,minimum height=20pt, minimum width=10pt,text width=10pt] (bt2) {\includegraphics{bluetooth}};
		
		\node[right=10pt of bt2] 			(huvudmodul) 	{Huvudmodul};
		\node[below=-10pt of huvudmodul,draw=none] (master) {\textit{master}};
		\node[right=10pt of huvudmodul] 		(sensormodul) 	{Sensormodul};
		\node[below=-10pt of sensormodul,draw=none] (slave1) {\textit{slav}};
		\node[right=10pt of sensormodul] 	(styrmodul) 		{Styrmodul};
		\node[below=-10pt of styrmodul,draw=none] (slave2) {\textit{slav}};
		
		\coordinate (sclStart) 	at ($(huvudmodul.north west) + (0,20pt)$);
		\coordinate (sclEnd)		at ($(styrmodul.north east)  + (0,20pt)$);
		
		\coordinate (sdaStart)  at ($(sclStart) + (0,20pt)$);
		\coordinate (sdaEnd)		at ($(sclEnd)	+ (0,20pt)$);
		
		\coordinate (vddStart)  at ($(sdaStart) + (0,40pt)$);
		\coordinate (vddEnd)		at ($(sdaEnd)	+ (0,40pt)$);
		
		\draw[thick] (sclStart) -- (sclEnd) node [right,draw=none,text width=0,minimum width=0] {SCL};
		\draw[thick] (sdaStart) -- (sdaEnd) node [right,draw=none,text width=0,minimum width=0] {SDA};
		\draw[thick] (vddStart) -- (vddEnd) node [right,draw=none,text width=0,minimum width=0] {$V_{dd}$};
		
		\draw[thick] (datormodul.east) -- (bt1.west);
		
		\draw[thick, ->,line join=round,decorate, decoration={
    												snake,
    												segment length=5,
    												amplitude=1,
    												post=lineto,
    												post length=1pt}] 
    		($(bt1.east) + (5pt,5pt)$) -- ($(bt2.west) + (-5pt,5pt)$);
    		
    	\draw[thick, ->,line join=round,decorate, decoration={
    												snake,
    												segment length=5,
    												amplitude=1,
    												post=lineto,
    												post length=1pt}] 
    		 ($(bt2.west) + (-5pt,-5pt)$) -- ($(bt1.east) + (5pt,-5pt)$);
    		 
    	\draw[thick] (bt2.east) -- (huvudmodul.west);
    	
    	\draw[thick,fill=black] ($(huvudmodul.north) + (-10pt,0)$) -- ($(huvudmodul.north) + (-10pt,20pt)$) circle [radius=2pt];
    	\draw[thick,fill=black] ($(huvudmodul.north) + (+10pt,0)$) -- ($(huvudmodul.north) + (+10pt,40pt)$) circle [radius=2pt];
    	
    	\draw[thick,fill=black] ($(sensormodul.north) + (-10pt,0)$) -- ($(sensormodul.north) + (-10pt,20pt)$) circle [radius=2pt];
    	\draw[thick,fill=black] ($(sensormodul.north) + (+10pt,0)$) -- ($(sensormodul.north) + (+10pt,40pt)$) circle [radius=2pt];
    	
    	\draw[thick,fill=black] ($(styrmodul.north) + (-10pt,0)$) -- ($(styrmodul.north) + (-10pt,20pt)$) circle [radius=2pt];
    	\draw[thick,fill=black] ($(styrmodul.north) + (+10pt,0)$) -- ($(styrmodul.north) + (+10pt,40pt)$) circle [radius=2pt];
    	
    	\draw[thick,fill=black] ($(sensormodul.north east) + (-5pt,20pt)$) circle [radius=2pt] -- ($(sensormodul.north east) + (-5pt,80pt)$) circle [radius=2pt];
    	\draw[thick,fill=black] ($(styrmodul.north west) + (5pt,40pt)$) circle [radius=2pt] -- ($(styrmodul.north west) + (5pt,80pt)$) circle [radius=2pt];
    	\draw[thick,draw=black,fill=white] ($(styrmodul.north west) + (2pt,50pt)$) rectangle ($(styrmodul.north west) + (8pt,70pt)$);
		\draw[thick,draw=black,fill=white] ($(sensormodul.north east) + (-8pt,50pt)$) rectangle ($(sensormodul.north east) + (-2pt,70pt)$) node [below left=-10pt and 20pt,draw=none,minimum width=0, text width = 0pt] {$R_p$};
		
		\draw[thick, ->] ($(sensormodul.south) + (+40pt,0)$) -- ($(sensormodul.south) + (40pt,-30pt)$) -- ($(huvudmodul.south) + (40pt,-30pt)$) -- ($(huvudmodul.south) + (40pt,0)$);
		\draw[thick, ->] ($(styrmodul.south) + (+40pt,0)$) -- ($(styrmodul.south) + (40pt,-40pt)$) -- node [midway,below,minimum height=20pt, draw=none] {Avbrott}($(huvudmodul.south) + (30pt,-40pt)$) -- ($(huvudmodul.south) + (30pt,0)$);
	\end{tikzpicture}
\end{document}	}
	\caption{Intermodulär kommunikation \label{communication}}
\end{figure}

\pagebreak
\section{Delsystem 1 - Huvudmodul}
\label{sec:huvudmodul}

Genom delsystem 1, tillika systemets huvudmodul, så kommer all komunikation mellan robototen och en extern datormodul att ske. Systemkonstruktionen för roboten är konstruerad på ett sådant vis att huvudmodulen kommer att verka som överordnad processorenhet (Master) i den tvåtrådsbuss som förbinder robotens olika delsystem. 
\subsection{Strömrytare}
Robotens strömbrytare ska vara kopplad till en av huvudmodulens avbrottsingång. När denna triggas så ska processorerna i huvudmodulen, sensormodulen samt styrmodulen gå in i viloläge. Anledningen till att dessa ska inträda viloläge, snarare än att bryta strömförsörjningen helt, är främst för att bespara användaren att ånyo tvingas kalibrera sensorerna vid varje uppstart.
\subsection{Brytare för autonomt respektive manuellt läge}
Brytaren mellan autonomt och manuellt körläge tillåter användaren att välja mellan robotens två körlägen. Liksom strömbrytaren så ska även denna kopplas till en av huvudmodulens avbrottsingångar samt vara studsfri för att säkerställa systemet ingår det körläge som användaren önskat.
\subsection{Bluetooth\textsuperscript{\circledR} sändare/mottagare}
Kommunikationen mellan den externa datormodulen och roboten kommer att ske via Bluetooth\textsuperscript{\circledR}. Det data som sensormodulen har samlat in (och bearbetat) kommer att kommuniceras till huvudenheten, vilken kommer att vidarebefordra mätdata från banan till datormodulen. Datormodulen ska därefter rita upp en karta baserad på det mätdata som den mottagit. Vid manuellt körläge så ska huvudmodulen i sin tur mottaga styrkommandon från datormodulen 
 

\pagebreak
\section{Delsystem 2 - Sensormodul}
\label{sec:sensormodul}
Sensormodulens uppdrag är att kommunicera med både sensorerna och huvudmodulen. Mätdata samplas med konstant frekvens och konverteras därefter till motsvarande SI-enhet. För avståndssensorerna innebär detta att den analoga inspänningen konverteras till ett digitalt värde i enheten meter.

Vinkelhastighets-sensorerna används för att beräkna vinkelutslag då roboten tar en sväng. Mätdatan behöver därför integreras under ett tidsintervall och konverteras till en vinkel. Hur den nödställde ska representeras är ännu inte fastställt. Förslagsvis används en IR-fyr i kombination med en IR-sensor. Sensormodulens uppdrag blir då att ta beslutet om när den nödställde är funnen.


\subsection{Sensorer}
De sensorer som används är följande,
\begin{description}
	\item[Avstånd] \hfill \\
	4 x IR-sensor \verb+GP2D120+ (4 cm till 30 cm) \\
	1 x Laser-sensor \verb+LIDAR-Lite v2+ (0 till 40 m) \\
	\item[Vinkelhastighet] \hfill \\
	1 x Gyro/accelerometer \verb+MPU-6500+ 
	\item[Identifierare av nödställd] \hfill \\
	1 x IR-detektor \verb+IRM-8601-S+
\end{description}
och är placerade enligt figur \ref{sensors}.  

\subsection{Kommunikation med huvudmodul}
Sensormodulen kommunicerar med huvudmodulen enligt den I\textsuperscript{2}C-buss som är definierad i kapitel \ref{sec:huvudmodul}. Eftersom sensormodulen är \emph{slave} i sammanhanget så anropar den huvudmodulen via ett avbrott när information vill skickas. Informationen som skickas är avståndet till den främre väggen och vinkeln mot väggen, se figur \ref{distances}. 
\pagebreak
\section{Delsystem 3 - Styrmodul}
Följande delar ska ingå i styrmodulen.
\subsection{Motorer}

Det chassi som roboten byggs på har fyra hjul. Dessa styrs parvis, höger hjulpar respektive vänster hjulpar. Signalerna till respektive sida styr rotationsriktning respektive hastighet. Signalen som styr hastigheten ska vara pulsbreddsmodulerad.
\subsection{LCD-display}
För att skriva ut sensordata på display används en alphanumerisk LCD-display. Förslaget är att använda den alfanumeriska displayen LCD JM162A. Den har totalt 16 pinnar, där 8 stycken är databitar, 3 stycken för konfiguration (inställning av funktion) och resterande är referenssignaler och strömtillförsel.

\subsection{Gripklo}
Gripklon ska kunna styras för att öppna och stänga. 

\pagebreak
\section{Delsystem 4 - Datormodul}
Följande delar återfinns i datormodulen.
\subsection{Bluetooth\textsuperscript{\circledR} sändare/mottagare}
\textit{Se avsnitt ......}
\subsection{Gränssnitt}
GUI. Färdigt program? Programmeringsspråk? Vad ska användaren kunna göra? Vad ska synas?

\setcounter{secnumdepth}{0}
\pagebreak
\begin{thebibliography}{9}

\bibitem{krav}
  Grupp 4,
  \emph{Kravspecifikation 1.0},
  2016-02-03
  
\bibitem{banspec}
	Grupp 1-4,
	\emph{Banspecifikation 1.0}
	
\bibitem{tavling}
	Grupp 1-4,
	\emph{Tävlingsregler 1.0}

\end{thebibliography}

\setcounter{secnumdepth}{2}

\end{flushleft}

\end{document}